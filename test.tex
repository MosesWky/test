\documentclass{ctexart}

\usepackage{Standard1}
\usepackage{geometry}
%\geometry{a4paper,scale=0.7}
\geometry{a4paper,left=2.5cm,right=2.5cm,top=2.5cm,bottom=2.5cm}
\lhead{}\rhead{pde回忆卷(部分)}
\pagenumbering{arabic}
\lfoot{}%这条语句可以让页码出现在下方
\setlength\headwidth{\textwidth}%让页眉宽度与正文一致
\usepackage{amsmath}
\usepackage{mathtools}
\usepackage{graphicx}

\usepackage[super]{gbt7714}

\usepackage{xcolor}
\usepackage{makecell}
\usepackage{fancybox}
\usepackage{pdfpages}
%\usepackage{fdsymbol}
\usepackage[all,pdf]{xy}

\usepackage{tikz}  
\usetikzlibrary{arrows,shapes,chains}  

\begin{document}

%\songti

%\pagestyle{empty}

%\zihao{3}
\noindent\textbf{A组}

6.求出满足极小曲面方程
\[(1+u_y^2)u_{xx}-2u_xu_yu_{xy}+(1+u_x^2)u_{yy}=0.\]
的所有具有$u=f(\sqrt{x^2+y^2})$形式的极小曲面.

7.求解由下述Laplace方程的第一边值问题所描述的矩形平板($0\le x\le a,0\le y\le b$)上的稳定温度分布:
\[\begin{cases}
	u_{xx}+u_{yy}=0,\\
	u(0,y)=u(a,y)=0,\\
	u(x,0)=\sin\frac{\pi x}{a},u(x,b)=0.
\end{cases}\]



















%\noindent
%\doublebox{
%	\parbox{0.96\textwidth}{{\songti\small
%		\textbf{[计算复杂性]} 二、某汽车公司生产 $k$ 种型号的汽车, 某天计划装配型号为 $j, j=1, \cdots, k$ 的汽车 $n_j$ 辆。装配作业种类计有 $m$ 种, 不同型号的汽车所需进行的作业种类可能不同。若 型号为 $j, j=1, \cdots, k$ 的汽车需要经过作业 $i, i=1, \cdots, m$, 则记 $a_{i j}=1$, 否则记 $a_{i j}=0$ 。 所有拟于当天装配的各种型号的汽车排成一列依次经过生产线, 该顺序一经确定不可在装配进行过程中更改。规定对作业 $i, i=1, \cdots, m$, 经过生产线的任意连续 $s_i$ 辆汽车中至多只能有 $r_i$ 辆汽车需要该项作业。\textbf{车辆顺序问题(car sequencing)} 要求对给定的上述参数, 判断是否存在当天装配汽车的一种可行排列顺序。\\
%		(1) 有同学认为, 非确定性算法可猜想出一种排列顺序,并验证该顺序是否符 合每种作业对间隔的要求, 因此该问题属于 $\mathcal{NP}$。以上断言是否准确, 为什么? 若不准确, 试给出一条件, 使得在此条件下该问题属于 $\mathcal{NP}$;\\
%		(2) 试写出一数学规划, 用于判断可行排列顺序是否存在, 并在存在时给出其中一种;\\
%		(3) 证明: 即使对所有 $j, j=1, \cdots, k, n_j \equiv 1$, 且对所有作业 $i, i=1, \cdots, m$, 均有 $s_i=2, r_i=1$, 车辆顺序问题仍是 $\mathcal{NP}$-难的。(提示: 用 $\mathcal{NP}$-完全问题 Hamilton 路归约, 图的每个顶点对应一种型号。)}
%	}
%}
%
%\noindent 解 (1) 不准确。考虑输入规模是 $O\left(\max\limits_{j=1,\cdots,k}\log n_j\right)$, 然而验证一个排列顺序的复杂度是 $O\left(\sum\limits_{j=1}^k n_j\right)$, 这相对于 输入规模是指数级的, 因此不能说该问题属于 $\mathcal{NP}$。\par 
%根据上述讨论, 当 $n_1=\cdots=n_k=1$ 时, 我们可以认为该问题属于 $\mathrm{NP}$ 。\\
%(2) 记车的总数为 $N=\sum_{j=1}^k n_j$, 我们用一个 $N \times k$ 的 01 矩阵 $P$ 来表示车辆顺序, $P_{i j}=1$ 表示第 $i$ 辆车 型号为 $j$, 数学规划表示如下:
%$$
%\begin{array}{rlll}
%	\max _P & 1 \\
%	\text { s.t. } & \sum\limits_{c=t}^{t+s_i-1} \sum\limits_{j=1}^k P_{c j} a_{i j} \leq r_i & i=1, \ldots, m \quad t=1, \ldots, N-s_i+1 \\
%	& P_{i j} \in\{0,1\} & i=1, \ldots, N \quad j=1, \ldots, k \\
%	& \sum\limits_{j=1}^k P_{i j}=1 & i=1, \ldots, N  \\
%	& \sum\limits_{i=1}^N P_{i j}=n_j & j=1, \ldots, k 
%\end{array}
%$$\\
%(3) 考虑无向图 $G=(V, E)$ 的 Hamilton 路径问题, 这个问题是 $\mathrm{NP}$-完全的, 我们将其规约到特殊形式的车辆顺序问题。\\
%设 $V=\{1, \ldots, k\}$, 设 $E^{\prime}$ 为 $V$ 构成的简单完全图的边集, 记 $E^{\prime}-E=\left\{\left(j_{i, 1}, j_{i, 2}\right): i=1, \ldots, m\right\}$, 令
%$$
%a_{i t}=\left\{\begin{array}{ll}
%	1, & t=j_{i, 1} \text { 或 } j_{i, 2} \\
%	0, & \text { 其它 }
%\end{array} \quad i=1,2, \ldots, m\right.
%$$
%则在这个车辆顺序问题中, 需要满足的约束就是: $j_{i, 1}$ 与 $j_{i, 2}$ 号车不能相邻 $(i=1, \ldots, m)$ 。这与求 $G$ 的 Hamilton 路径等价, 完成规约, 故该特殊形式的车辆顺序问题是 $\mathrm{NP}$-难的。
%
%\vspace{1cm}\noindent
%\doublebox{
%	\parbox{0.96\textwidth}{{\songti\small
%			\textbf{[最优算法与近似算法]}一、现有 $n$ 个工件需安排在一台机器上加工, 工件 $J_j$ 的加工时间为 $p_j, j=1, \cdots, n$ 。 机器在所有工件加工完毕前不可空闲, 工件加工不可中断。目标为工件完工时间 的方差尽可能小, 即 $S=\sum\limits_{j=1}^n\left(C_j-\bar{C}\right)^2$ 最小, 其中 $C_j$ 为工件 $J_j$ 的完工时间, $\bar{C}=\frac{1}{n} \sum\limits_{j=1}^n C_j 。$\\
%			(1) 若工件按照 $J_1, J_2, \cdots, J_n$ 的顺序依次加工, 试写出 $S$ 的表达式 (用 $p_1, p_2, \cdots, p_n$ 表示);\\
%			(2) 证明: 存在一个最优解, 其中最早加工的工件是加工时间最大的工件;\\
%			(3) 设 $n=4, p_1 \geq p_2 \geq p_3 \geq p_4$, 试求出该问题最优解。
%			}
%	}
%}
%
%\noindent 解 (1) 简单计算如下:
%$$
%\begin{aligned}
%	C_j & =\sum_{i=1}^j p_j, \quad \bar{C}=\sum_{i=1}^n(1+\frac{1-i}{n}) p_i \\
%	S & =\sum_{j=1}^n\left(C_j-\bar{C}\right)^2 =\sum_{j=1}^n C_j^2-n \bar{C}^2 \\
%	& =\sum_{i=1}^n\left(\sum_{j=1}^i p_j\right)^2-n\left(\sum_{i=1}^n(1+\frac{1-i}{n}) p_i\right)^2 
%\end{aligned}
%$$
%(2) 设有个最优解$\Gamma$,按顺序加工时间依次为$r_1,\cdots,r_n$(易知其为$p_1,\cdots,p_n$的一个排列)。则 $$S=\sum_{i=1}^n\left(\sum_{j=1}^i r_j\right)^2-n\left(\sum_{i=1}^n(1+\frac{1-i}{n}) r_i\right)^2 $$
%可知$r_1^2$系数为$n-n=0$,$r_1r_j(j\ne 1)$系数为$2(n+1-j)-2n(1+\frac{1-j}{n})=0$。\\
%即$S=f(r_2,\cdots,r_n)$与$r_1$无关且易知其各项系数均为正($\star$)\\
%若$r_1=\max\limits_{1\le i\le n}p_i$,则$\Gamma$满足要求。反之,设$r_\alpha=\max\limits_{1\le i\le n}p_i>r_1(\alpha\ne 1)$。将$r_\alpha$与$r_1$位置对调得$\Gamma'$,由($\star$)知$S'<S$。这与$\Gamma$的最优性矛盾,故原命题成立。\\
%(3) 由(2),考虑$J_1$最早加工,设加工顺序为$J_1,J_a,J_b,J_c$,其中$\{a,b,c\}=\{2,3,4\}$,记$M=p_2+p_3+p_4$,则 
%\[\begin{split}
%	S &= p_1^2+(p_1+p_a)^2+(p_1+p_a+p_b)^2+(p_1+p_a+p_b+p_c)^2-4(p_1+\frac 34p_a+\frac 12p_b+\frac 14p_c)^2\\
%	&= \frac 34p_a^2+p_b^2+\frac 34p_c^2+p_ap_b+\frac 12p_ap_c+p_bp_c\\
%	&= \frac 12(p_a+p_b)^2+\frac 12(p_b+p_c)^2+\frac 14(p_a+p_c)^2\\
%	&= \frac 12(M-p_c)^2+\frac 12(M-p_a)^2+\frac 14(M-p_b)^2.
%\end{split}\]
%
%由平方项前的权重可知$b=4,\{a,c\}=\{2,3\}$即可。即最优解有$J_1J_2J_4J_3$或$J_1J_3J_4J_2$,对应$S=\frac 12(p_2+p_4)^2+\frac 12(p_3+p_4)^2+\frac 14(p_2+p_3)^2.$
%
%\vspace{1cm}\noindent
%\doublebox{
%	\parbox{0.96\textwidth}{{\songti\small
%			\textbf{[最优算法与近似算法]}二、假设 $I$ 是 $n$ 个城市的度量 TSP 问题实例。城市 $i$ 和城市 $j$ 之间的距离为 $c_{i j}, i, j=1, \cdots, n 。 T^*$ 为实例 $I$ 的最优环游, 其长度为 1 。对任意城市 $i, j$, 记 $o_{i j}$ 为 $T^*$ 中自 $i$ 至 $j$ 部分的长度。定义函数
%			$$
%			d(x, y)=\min \{|x-y|, 1-|x-y|\},
%			$$
%			区域
%			$$
%			S(p, q, u, v)=\left\{(x, y) \in[0,1) \times[0,1) \mid d\left(x, o_{p u}\right)+d\left(y, o_{q v}\right)<c_{u v}\right\} 。
%			$$
%			(1) 求区域 $S(u, v, u, v)$ 的面积。
%			对 $I$ 的某一环游 $T: i_1 i_2 \cdots i_n$, 任取 $k \geq 1, k+2 \leq l \leq n$, 环游
%			$$
%			T^{\prime}: i_1 i_2 \cdots i_{k-1} i_k i_l i_{l-1} i_{l-2} \cdots i_{k+1} i_{l+1} i_{l+2} \cdots i_n
%			$$
%			称为 $T$ 的 2-change 环游, 即用 $i_k i_l$ 与 $i_{k+1} i_{l+1}$ 两条边代替 $T$ 中 $i_k i_{k+1}$ 与 $i_l i_{l+1}$ 两条边, 并 将 $T$ 中 $i_{k+1} i_{k+2} \cdots i_l$ 之间的边反向后得到的环游。环游长度不大于它的所有 2-change 环游的长度的环游称为 2-opt 环游。\\
%			(2) 试给出 2-opt 环游所具有的性质, 并证明: 若 $T$ 为 2-opt 环游, 则对任意固 定的城市 $p, q$, 对任意的 $1 \leq k<l \leq n$, 区域 $S\left(p, q, i_k, i_{k+1}\right)$ 与 $S\left(p, q, i_l, i_{l+1}\right)$ 的交为空 集。
%			度量 TSP 问题的 2-opt 算法是一种局部搜索算法, 它从任一环游开始, 若当 前环游不是 2-opt 环游, 则将其改进为它的任意一个长度更短的 2-change 环游, 直至当前环游为 2-opt 环游为止。
%		}
%	}
%}
%
%\newpage
%\noindent
%\doublebox{
%	\parbox{0.96\textwidth}{{\songti\small
%			(3) 证明: 对任意固定的城市 $u, v$, 对任意的城市 $p_1, q_1, p_2, q_2, S\left(p_1, q_1, u, v\right)$ 的 面积与 $S\left(p_2, q_2, u, v\right)$ 的面积相等。\\
%			(4) 证明: 2-opt 算法求解 $n$ 个城市的度量 TSP 问题的最坏情况比不超过 $\sqrt{\dfrac{n}{2}}$ 。
%		}
%	}
%}
%
%\noindent 解 (1) $S(u,v,u,v)$ 的形状如下面绿色区域所示。
%
%\begin{figure}[ht]
%	\centering
%	\includegraphics[width=0.2\linewidth]{figs/hw5-1}
%\end{figure}
%
%注意到 $c_{ij}$ 具有度量性质,因此最优环游的长度至少为 $2c_{uv}$ ,对任意城市$u,v$成立。从而我们有 $c_{uv}\le \frac 12$,因此四个三角形不会出现相交。于是 $$Area S(u,v,u,v)=2c_{uv}^2$$
%(2) 对于一个 2-opt 环游,做一次 2-change 得到的新环游总不会更优。
%
%\begin{figure}[ht]
%	\centering
%	\includegraphics[width=0.8\linewidth]{figs/hw5-2}
%\end{figure}
%
%下面来证明结论: $S\left(p, q, i_k, i_{k+1}\right) \cap S\left(p, q, i_l, i_{l+1}\right)=\varnothing$.\\
%用反证法, 假设 $(x, y)$ 是它们交集中的一点。注意到 $d$ 满足度量性质,有三角不等式:
%$$
%d(a, b)+d(b, c) \leq d(a, c), \quad \forall a, b, c \in[0,1)
%$$
%再根据度量 TSP 应满足的度量性质,有:
%$$
%\begin{aligned}
%	c_{i_k i_l}+c_{i_{k+1} i_{l+1}} & \leq d\left(o_{p i_k}, o_{p i_l}\right)+d\left(o_{q i_{k+1}}, o_{q i_{l+1}}\right) \\
%	& \leq d\left(o_{p i_k}, x\right)+d\left(x, o_{p i_l}\right)+d\left(o_{q i_{k+1}}, y\right)+d\left(y, o_{q i_{l+1}}\right) \\
%	& <c_{i_k i_{k+1}}+c_{i_l i_{l+1}}
%\end{aligned}
%$$
%于是对图 2 所示的 2-opt 环游进行图 3 所示的 2-change 之后得到的新环游严格更优, 矛盾。\\
%(3) 这是显然的, 因为 $S\left(r_1, r_2, u, v\right)$ 的形状拼起来总是一个正方形, $r_1, r_2$ 只能决定正方形中点的位置, 其总 是以 $c_{u v}$ 为一半对角线的长度。因此 $r_1, r_2$ 不能改变面积。\\
%(4) 根据 $(1)(2)(3)$ 的结论, 有
%$$
%\sum_{k=1}^n c_{i_k i_{k+1}}^2=\frac{1}{2} \sum_{k=1}^n S\left(p, q, i_k, i_{k+1}\right) \leq \frac{1}{2}
%$$
%根据柯西不等式有:
%$$
%l(T)=\sum_{k=1}^n c_{i_k i_{k+1}} \leq \sqrt{n \sum_{k=1}^n c_{i_k i_{k+1}}^2} \leq \sqrt{\frac{n}{2}}
%$$
%即得证。
%
%\includepdf[pages=-]{辅助/组合优化习题.pdf}

%\newpage
%\section{概率论作业整理}
%\subsection{第一章}
%
%\noindent T15.在一张打方格的纸上投一枚直径为$1$的硬币, 问方格要多小才能使硬币与线不相交的概率小于$1\%$?
%
%\vspace{2cm}
%\noindent T25.考试时共有$n$张考签, $m$($m\ge n$)个学生参加考试, 被抽过的考签立即放回, 求在考试结束后, 至少有一张考签没有被抽到的概率.
%
%\vspace{2cm}
%\noindent T34.某厂有甲、乙、丙三台机器生产螺丝钉, 产量各占 $25 \%, 35 \%, 40 \%$; 在各自的产品里, 不合格品各占 $5 \%, 4 \%, 2 \%$.\\
%(1) 从产品中任取一只, 求它恰是不合格品的概率;\\
%(2) 若任取一只恰是不合格品, 求它是机器甲生产的概率.
%
%\vspace{3cm}
%\noindent T55.每个蚕产$k$个卵的概率为$\dfrac{\lambda^ke^{-\lambda}}{k!}$, 其中$\lambda>0$为常数. 而每个卵能变为成虫的概率为$p$, 各卵是否变为成虫相互独立. 求证: 每蚕养出$r$个小蚕的概率为$\dfrac{(\lambda p)^re^{-\lambda p}}{r!}$.
%
%\vspace{3cm}
%
%%\newpage
%\subsection{第二章}
%
%\noindent T9.一本$500$页的书中共有$500$个错误, 每个错误等可能地出现在每一页上, 求指定一页上至少有$3$个错误的概率.
%
%\vspace{2cm}
%\noindent T27.$\xi\sim U[0,5]$, 求方程$4x^2+4\xi x+\xi+2=0$有实根的概率.
%
%\vspace{2cm}
%\noindent T31.若 $(\xi, \eta)$ 的密度函数为
%\[p(x, y)= \begin{cases}A e^{-(2 x+y)}, & x, y>0, \\ 0, & \text { 其他. }\end{cases}\]
%求\\
%(1) 常数 $A$;
%(2) 分布函数 $F(x, y)$;
%(3) $\xi$ 的边际密度;
%(4) $P(\xi<2,0<\eta<1)$;
%(5) $P(\xi+\eta<2)$;
%(6) $P(\xi=\eta)$.\\
%T.33. 条件密度$p_{\eta\vert\xi}(y\vert x)$
%
%\vspace{5cm}
%\noindent T41.设 $(\xi, \eta, \zeta)$ 的联合密度函数为
%\[p(x, y, z)= \begin{cases}\dfrac{1-\sin x \sin y \sin z}{8 \pi^3}, & 0 \leq x, y, z \leq 2 \pi, \\ 0, & \text { 其它. }\end{cases}\]
%求证 $\xi, \eta, \zeta$ 两两独立, 但不相互独立.
%
%\vspace{4cm}
%\noindent T50.设$\xi\sim N(a,\sigma^2)$, 求$e^{\xi}$的密度函数(称为对数正态分布)
%
%\vspace{2cm}
%\noindent T59.设火炮射击时弹着点坐标 $(\xi, \eta)$ 服从二维正态分布 $N\left(0,0, \sigma^2, \sigma^2, 0\right)$, 求距离 $\rho=$ $\sqrt{\xi^2+\eta^2}$ 的分布密度.
%
%\vspace{3cm}
%\noindent T61.$(\xi, \eta)$ 的联合密度为
%\[p(x, y)= \begin{cases}4 x y, & 0<x, y<1, \\ 0, & \text { 其他. }\end{cases}\]
%求 $\left(\xi^2, \eta^2\right)$ 的联合密度.
%
%\vspace{3cm}
%\noindent T63.设 $(\xi, \eta)$ 服从二元正态分布 $N\left(0,0, \sigma_1^2, \sigma_2^2, r\right)$. 求 $\xi+\eta$ 与 $\xi-\eta$ 相互独立的充要条件.
%
%\vspace{5cm}
%
%%\newpage
%\subsection{第三章}
%
%\noindent T9.设$\xi_1,\xi_2$相互独立, 均服从$N(a,\sigma^2)$. 求证: \[E\max (\xi_1,\xi_2)=a+\dfrac{\sigma}{\sqrt{\pi}}.\]
%
%\vspace{3cm}
%\noindent T24.设$\xi$只取值于$[a,b]$, 求证: $Var\xi\le \dfrac{(b-a)^2}{4}$.
%
%\vspace{2cm}
%\noindent T36.设 $\xi, \eta$ 都是只取两个值的随机变量, 求证: 如果它们不相关, 则它们独立.
%
%\vspace{4cm}
%\noindent T38.设随机变量 $\xi_1, \cdots, \xi_n$ 中任意两个相关系数都是 $\rho$, 求证: $\rho \geq-1 /(n-1)$.
%
%\vspace{3cm}
%\noindent T39.求下列分布的特征函数:\\
%(5) $\xi$ 的密度为
%\[p(x)= \begin{cases}\frac{2+x}{4}, & -2 \leq x<0, \\ \frac{2-x}{4}, & 0 \leq x<2, \\ 0, & \text { 其他; }\end{cases}\]
%(6) $\eta=a \xi+b$, 其中 $\xi$ 服从 $[0,1]$ 上的均匀分布;
%
%\vspace{5cm}
%\noindent T42.证明下列函数是特征函数, 并找出相应的分布.\\
%(3) $(\sin t / t)^2$\\
%(5) $\left(1+t^2\right)^{-1}$
%
%\vspace{5cm}
%\noindent T45.证明
%\[\varphi(t)=\left\{\begin{array}{ll}
%	1-\frac{|t|}{a}, & |t|<a, \\
%	0, & |t| \geq a
%\end{array} \quad a>0\right.\]
%是特征函数, 并求出对应的分布.
%
%\vspace{3cm}
%
%%\newpage
%\subsection{第四章}
%
%\noindent T2.设 $\xi_n$ 的分布列为: $P\left(\xi_n=0\right)=1-1 / n, \quad P\left(\xi_n=n\right)=1 / n, n=1,2, \cdots$. 求证: 相应 的分布函数列收敛于分布函数, 但 $E \xi_n$ 不收敛于相应分布的期望.
%
%\vspace{3cm}
%\noindent T6.某车间有 $200$ 台车床, 工作时每台车床 $60 \%$ 时间在开动, 每台开动时耗电1千瓦. 问: 应供给这个车间多少电力才能有$0.999$ 的把握保证正常生产?
%
%\vspace{3cm}
%\noindent T10.设 $\left\{\xi_n\right\}$ 独立同分布, 其分布分别为 (1) $[-a, a]$ 上的均匀分布; (2) 泊松分布. 记\[\eta_n=\sum\limits_{k=1}^n\left(\xi_k-E \xi_k\right) / \sqrt{\sum\limits_{k=1}^n \operatorname{Var} \xi_k}\]计算 $\eta_n$ 的特征函数, 并求 $n \rightarrow \infty$ 时的极限. 从而验证林德贝格-勒维定理在这种情况成立.
%
%\vspace{5cm}
%\noindent T14.设 $\left\{\xi_n\right\},\left\{\eta_n\right\}$ 各为独立同分布随机变量序列, 它们之间相互独立. $E \xi_n=0, \operatorname{Var} \xi_n=1$, $P\left(\eta_n=\pm 1\right)=1 / 2$. 求证: $S_n=\sum\limits_{k=1}^n \xi_k \eta_k / \sqrt{n}$ 的分布函数收敛于 $N(0,1)$.
%
%\vspace{3cm}
%\noindent T19.求证下列各独立随机变量序列 $\left\{\xi_k\right\}$ 服从大数定律.\\
%(1) $P\left(\xi_k=\sqrt{\ln k}\right)=P\left(\xi_k=-\sqrt{\ln k}\right)=\dfrac 12$;\\
%(4) $P\left(\xi_k=n\right)=\dfrac{c}{n^2 \ln ^2 n}, n=2,3, \cdots, c$ 为常数.
%
%\vspace{6cm}
%\noindent T23.设 $\left\{\xi_k\right\}$ 独立同分布, 都服从 $[0,1]$ 上的均匀分布, 令 $\eta_n=\left(\prod\limits_{k=1}^n \xi_k\right)^{\frac 1n}$. 求证: $\eta_n \stackrel{P}{\longrightarrow} c$ (常数), 并求出 $c$.
%
%\vspace{3cm}
%\noindent T29.设 $\left\{\xi_k\right\}$ 独立同分布, 都服从 $N(0,1)$ 分布. 求证:
%$$
%\eta_n=\frac{\xi_1+\cdots+\xi_n}{\sqrt{\xi_1^2+\cdots+\xi_n^2}}
%$$
%渐近标准正态分布.


%\pagestyle{empty}
%\section{浙江大学2020年数学九推考试试题}
%\subsection{数学分析}
%\begin{enumerate}
%	\item (15 分)
%	\begin{enumerate}[(1)]
%		\item 求极限$\lim\limits_{x \rightarrow 0} \dfrac{x-\int_0^x \mathrm{e}^{t^2} \mathrm{~d} t}{x^2 \sin 3 x}.$
%		\item 讨论函数项级数 $\sum\limits_{n=1}^{\infty} \dfrac{(-1)^n}{1+n x}$ 在区间 $[a,+\infty)(a>0)$ 和 $[0,+\infty)$ 上的一致收敛性.
%	\end{enumerate}
%	\item (10 分) 应用条件极值的方法证明不等式
%	$$
%	\frac{1}{n} \sum_{k=1}^n x_k^2 \geqslant\left(\frac{1}{n} \sum_{k=1}^n x_k\right)^2, x_k \in \mathbb{R}, \forall 1 \leqslant k \leqslant n .
%	$$
%	\item (10 分) 设二元函数 $F$ 在 $\left(x_0, y_0\right)$ 的某邻域内二阶连续可微,
%	$$
%	F\left(x_0, y_0\right)=0, F_x^{\prime}\left(x_0, y_0\right)=0, F_y^{\prime}\left(x_0, y_0\right)>0, F_{x x}^{\prime \prime}\left(x_0, y_0\right)<0 .
%	$$
%	求证: 由方程 $F(x, y)$ 确定的在 $x_0$ 某邻域内的隐函数 $y=f(x)$ 在 $x_0$ 处达到局部极小值.
%	\item (15 分) 设 $f \in C^2(\mathbb{R})$, 其周期为 $2 \pi$, Fourier 系数为 $a_n, b_n$. 求证:
%	\begin{enumerate}[(1)]
%		\item $		a_n=O\left(\dfrac{1}{n^2}\right), b_n=O\left(\dfrac{1}{n^2}\right), n \rightarrow \infty .$
%		\item $f$ 的 Fourier 级数一致收敛到 $f$.
%		\item $\dfrac{a_0^2}{2}+\sum\limits_{n=1}^{\infty}\left(a_n^2+b_n^2\right)=\dfrac{1}{\pi} \int_{-\pi}^\pi|f|^2 .$
%	\end{enumerate}
%\end{enumerate}
%
%\subsection{高等代数}
%\begin{enumerate}
%	\item (10 分) 问: $\lambda$ 满足什么条件时, 复多项式 $x^3-3 x^2+\lambda x-1$ 有重因式?
%	\item (10 分) 设 $W_1, W_2$ 分别是 $\mathbb{R}$ 上线性方程组 $A X=0, B X=0$ 的解空间, $W_0$ 是 $\mathbb{R}$ 上线性方程组 $\left(\begin{array}{c}A \\ B\end{array}\right) X=0$ 的解空间, 其中 $X \in \mathbb{R}^n$. 求证: $W_0$ 在 $\mathbb{R}^n$ 上 的正交补空间等于 $W_1$ 和 $W_2$ 各自在 $\mathbb{R}^n$ 上的正交补空间之和.
%	\item (10 分) 设 $A \in M_{n \times n}(\mathbb{C})$.
%	\begin{enumerate}[(1)]
%		\item 给出 $A$ 的极小多项式 $m_A(x)$ 的定义.
%		\item 设 $A$ 的特征多项式为 $f_A(x)$. 求证: $m_A(x) \mid f_A(x)$.
%		\item 叙述一种求极小多项式的方法, 说明用这个方法的具体求解过程.
%	\end{enumerate}
%	\item (10 分) 求方阵 $A$ 的 Jordan 标准形 $J$, 并求可逆矩阵 $P$, 使得 $P^{-1} A P=J$, 其中
%	$$
%	A=\left(\begin{array}{ccc}
%		1 & 2 & -6 \\
%		1 & 0 & -3 \\
%		1 & 1 & -4
%	\end{array}\right) .
%	$$
%	\item (10 分) 设 $\mathbb{F}$ 为数域, $A \in M_{m \times n}(\mathbb{F})$. 问:矩阵方程组 $A X A=A, X A X=X$ 是否有非零解? 若有解, 请求出所有解; 若无解, 请说明理由.
%\end{enumerate}


\end{document}
